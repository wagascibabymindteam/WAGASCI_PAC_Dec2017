\subsection{Wagasci module}
The dimension of the central detector is 100cm $\times$ 100cm in the x and y directions
and 200cm along the beam direction.
The total water and hydrocarbon masses serving as neutrino targets are $\sim$1 ton each.
Inside the central detector, plastic scintillator bars are aligned as a 3D grid-like structure, shown in Fig. \ref{fig:3dgrid},
and spaces in the structure are filled with the neutrino target materials, water and hydrocarbon.
When neutrinos interact with hydrogen, oxygen or carbon, in water and hydrocarbon,
charged particles are generated.
Neutrino interactions are identified by detecting tracks of charged particles through plastic scintillation bars.
Thanks to the 3 D grid-like structure of the scintillator bars, 
the central detector has $4\pi$ angular acceptance for charged particles.
Furthermore, adopting a 2.5cm grid spacing, short tracks originated from protons and charged pions can be reconstructed
with high efficiency.
Thin plastic scintillator bars (thickness $\sim 0.3$cm) will be used for the central detector
to reduce  the mass ratio of scintillator bars to neutrino target materials,
because neutrino interactions in the scintillator bars are a background for the cross section measurements.
Scintillator bars whose dimensions are 2.5cm x 0.3cm x 100cm will be used for the central detector.
The total number of channels in the central detector is 12880.

\begin{figure}[tbh]
\begin{center}
\includegraphics[width=0.6\linewidth]{fig/3d_grid_detector.pdf}
\end{center}
\caption{
Schematic view of 3D grid-like structure of plastic scintillator bars inside the central detector.
}
\label{fig:3dgrid}
\end{figure}
