\subsection{Wagasci module}
The Wagasci module is a neutrino target detector consists of a stainless tank filled with 16 scintillator tracking planes immersed, where each plane is an array of 80 scintillator bars.
The 40 bars, called parallel scintillators, are placed perpendicularly to the beam, and the other 40 bars, called grid scintillators, are placed in parallel to the beam with grid structure.

The dimension of the each Wagasci module is 100cm $\times$ 100cm in the x and y directions
and 50cm along the beam direction.
Thin plastic scintillator bars (thickness $\sim 0.3$cm) are used for the Wagasci module
to reduce  the mass ratio of scintillator bars to water,
because neutrino interactions in the scintillator bars are a background for the cross section measurements.

Inside the Wagasci module, plastic scintillator bars are aligned as a 3D grid-like structure, shown in Fig. \ref{fig:3dgrid}.


Spaces in the 3D grid-like structure are filled with water for the water-in Wagasci module.
The total water mass serving as neutrino targets in the detector are $\sim$0.5 ton.

When neutrinos interact with hydrogen, oxygen or carbon, in water and scintillators,
charged particles are generated.
Neutrino interactions are identified by detecting tracks of charged particles through plastic scintillation bars.
Thanks to the 3 D grid-like structure of the scintillator bars, 
the Wagasci module has $4\pi$ angular acceptance for charged particles.
Furthermore, adopting a 5cm grid spacing, short tracks originated from protons and charged pions can be reconstructed
with high efficiency.

Scintillator bars whose dimensions are 2.5cm x 0.3cm x 100cm are used for the Wagasci module.
The total number of channels in one Wagasci module is 1280.

\begin{figure}[tbhp]
  \begin{center}
   \begin{subfigure}{0.48\textwidth}
     \includegraphics[width=\linewidth]{fig/3d_grid_structure.pdf}
    \end{subfigure}
  \begin{subfigure}{0.48\textwidth}
      \includegraphics[width=\linewidth]{fig/wagasci_mod.pdf}
    \end{subfigure}    
    \end{center}
  \caption{2D track reconstruction efficiency as a function of number of hits (left) and track angle (right).
  Here the track angle is the one reconstructed by the INGRID module.}
\label{fig:wmefficiency}
\end{figure}


\begin{figure}[tbh]
\begin{center}
\includegraphics[width=1.0\linewidth]{fig/3d_grid_structure.pdf}
% \includegraphics[width=0.6\linewidth]{fig/tmp.pdf}
\end{center}
\caption{
Schematic view of 3D grid-like structure of plastic scintillator bars inside the central detector.
}
\label{fig:3dgrid}
\end{figure}

\begin{figure}[tbh]
\begin{center}
\includegraphics[width=0.6\linewidth]{fig/wagasci_mod.pdf}
% \includegraphics[width=0.6\linewidth]{fig/tmp.pdf}
\end{center}
\caption{
Schematic view of Wagasci module.
}
\label{fig:wagasci_mod}
\end{figure}

\begin{figure}[tbh]
\begin{center}
\includegraphics[width=0.8\linewidth]{fig/wagasci_scinti_geometry.pdf}
% \includegraphics[width=0.6\linewidth]{fig/tmp.pdf}
\end{center}
\caption{
Geometry of scintillators used for Wagasci modules.
}
\label{fig:wagasci_scinti_geometry}
\end{figure}

