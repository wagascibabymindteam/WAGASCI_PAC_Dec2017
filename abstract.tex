\begin{abstract}
We, the WAGASCI collaboration, proposes to perform a study of neutrino-nucleus interactions on the B2 floor of the neutrino monitor building with a new-type fine-grained neutrino detector and muon range detectors.
 The hollow cuboid lattice scintillators filled with water as the neutrino interaction target (known as WAGASCI module) would enable the measurement of cross section
 on $\mathrm{H_2O}$.
Measurement in wide phase space becomes possible by the combination of the WAGASCI modules,
side- and downstream- muon range detectors (MRD's).
   %  , where different neutrino spectra from the T2K off-axis near detector (ND280) can be obtained due to the different off-axis position.
%Our experimental setup contains two hollow cuboid lattice detectors as the neutrino interaction target (known as WAGASCI modules), two side- and one downstream- muon range detectors(MRD’s).
The downstream-MRD, 
the so-called Baby MIND
% so called the Baby MIND 
detector, also works as a magnet 
and provides charge identification capability
% and provides the charge identification capability 
as well as magnetic momentum measurement for high energy muons.
The nominal experimental setup has two WAGASCI modules.
Most of the detectors have already been constructed and have been commissioned in the J-PARC T59 experiment and the CERN neutrino platform.
Therefore, the collaboration will be ready to collect physics data by January 2019.
% and the collaboration will be ready to proceed to the physics data taking by January 2019.
The experiment can run parasitically with T2K, without dedicated beam time.
With one-year data taking (roughly $5\times 10^{20}$ POT) in neutrino-mode and another one-year in antineutrino mode,
expected numbers of charged-current interaction event are 5,400 and 2,240 for one WAGASCI module respectively.
We will provide inclusive and exclusive differential cross sections of the charged current neutrino and antineutrino interactions with water and hydrocarbon
with a slightly higher neutrino energy than T2K ND280 with wider angler acceptance.
By combining our measurements with those from ND280, model-independent extraction of the cross section for narrow energy spread becomes possible.
These measurements would improve the understanding of the neutrino-nucleus interaction at around 1~GeV and also contribute to reducing one of the most significant uncertainties
of the T2K experiment.

\end{abstract}
