\section{Status of J-PARC T59 experiment}
We had submitted a proposal of a test experiment to test a new detector with a water target, WAGASCI, at the T2K near detector hall to J-PARC PAC on April 2014, and the proposal was approved as J-PARC T59.
 There are several updates on the project after three years from then.
 Fist, the start time of neutrino beam measurement is changed from December 2015 to October 2017, and the requested neutrino beam is changed from $1\times10^{21}$ POT of $\nu$ beam to $0.8\times10^{21} $POT of anti-$\nu$ beam. 
 Second, the detector configuration is changed. In the original proposal, central neutrino detector are expected to be surrounded by newly developed muon-range detectors (MRDs), but we will use spare neutrino detectors of the T2K experiment instead of them during neutrino beam measurement from October to December 2017. Construction of the newly developed MRDs, Baby-MIND and Side-MRD, is in progress, and they will be installed to the both sides and the downstream of the central neutrino detector from January to March 2018. Then, we will resume neutrino beam measurements from March 2018 and will take the neutrino beam data until May 2018.


\subsection{On-axis beam measurement with Prototype detector}
Add INGRID water module measurement here.


\subsection{Plans from October 2017 to May 2018}
J-PARC MR will extract its proton beam to T2K neutrino beam-line from October to December 2017, and, from March to May 2018. T2K experiment will produce anti-neutrino beam and will accumulate $\sim8\times10^{20}$ POT data during the above period.


 J-PARC T59 will perform neutrino beam measurements on the B2 floor of the T2K near neutrino detector hall during the above period to test basic performances of the WAGASCI detector and new electronics. During the beam measurements from October to December 2017, one WAGASCI module will be placed between spare neutrino detectors of the T2K experiment, INGRID Proton module and INGRID standard module as shown in Fig. \ref{fig:det_config_oct_dec2017}.
Detector location on the B2 floor of T2K near detector hall is shown in Fig. \ref{fig:det_loc_oct_dec2017}. Here, the INGRID Proton module is used as a charged particle VETO detector and, the INGRID standard module is used as a downstream muon detector. 
We had submitted a proposal to use these spare neutrino detectors for the T59 neutrino beam measurements to the T2K collaboration, and we got an approval from T2K. 

 \begin{figure}[tbh]
\begin{center}
\includegraphics[width=0.8\linewidth]{fig/t59_det_config_oct_dec_2017.pdf}
\end{center}
\caption{
J-PARC T59 detector configuration from Oct. to Dec. 2017
}
\label{fig:det_confg_oct_dec2017}
\end{figure}


\begin{figure}[tbh]
\begin{center}
\includegraphics[width=0.8\linewidth]{fig/t59_det_location_oct_dec_2017.pdf}
\end{center}
\caption{
J-PARC T59 detector location from Oct. to Dec. 2017
}
\label{fig:det_loc_oct_dec2017}
\end{figure}


During the beam measurements from March to May 2018, Baby-MIND and two side muon-range detector (Side-MRD) modules will be installed on the downstream and the both sides of the WAGASCI detector, as shown in Fig. \ref{fig:det_config_mar_may2018}, to increase angular acceptance for secondary charged particles from neutrino interactions.
Add Baby-MIND commissioning items here!!!
 
\begin{figure}[tbh]
\begin{center}
\includegraphics[width=0.8\linewidth]{fig/tmp.pdf}
% \includegraphics[width=0.8\linewidth]{fig/all_detector2.pdf}
\end{center}
\caption{
J-PARC T59 detector configuration with Baby-MIND and two Side-MRD modules from Mar. to May 2018.
(Need to prepare the figure.)}
\label{fig:det_config_mar_may2018}
\end{figure}


Expected number of neutrino events in the WAGASCI detector during the above beam period is evaluated with Monte Carlo simulations. 
Neutrino beam flux at the detector location is simulated by T2K neutrino flux generator, JNUBEAM, neutrino interactions with target materials are simulated by a neutrino interaction simulator, NEUT, detector responses are simulated using GEANT4-based simulation. 
The neutrino flux at the detector location, 1.5 degrees away from the J-PARC neutrino beam axis, is shown in Figure \ref{fig:b2flux}, and its mean neutrino energy is around 0.68 GeV.
An event display of the GEANT4-based detector simulation is shown in Figure \ref{fig:t59_event_display_oct_dec_2017}.

\begin{figure}[tbh]
\begin{center}
\includegraphics[width=0.8\linewidth]{fig/t59_event_display_oct_dec_2017.pdf}
% \includegraphics[width=0.8\linewidth]{fig/all_detector2.pdf}
\end{center}
\caption{
J-PARC T59 event display of a neutrino event in the GENAT4-based detector simulation.
}
\label{fig:t59_event_display_oct_dec_2017}
\end{figure}


To perform the detector performance test, the following event selections are applied to the data. 
First, track reconstructions are performed in the WAGASCI detector, and the reconstructed vertex is required to be inside a defined fiducial volume, $80 \times80 \times 32$ cm$^{3}$ region at the center of the detector, to reduce contamination from external backgrounds. 
Second, at least one charged particle is required to reach to INGRID standard module or Side-MRD modules, and it makes more than two hits in these sub-detectors. 
With the event selection, expected numbers of the neutrino-candidate events during the beam period are summarized in Table 1. 
Using the data, we will test the detector performance with $\sim3$\% statistical uncertainties.

