\section{Experimental Setup}
\textcolor{red}{This is copied from T59 proposal. Should be updated.}
Fig. \ref{fig:all_detector} shows a schematic view of the entire set of detectors.
A central detector contains the neutrino target materials, water and hydrocarbon, and plastic scintillator bars,
and is placed along the beam direction.
% The muon range detectors (MRDs) consist of one detector in the side region and one detector in the downstream region
% around the central detector.
The muon range detectors (MRDs) consist of one to four detectors in the side region and one detector in the downstream region
around the central detector.
% The number of the MRDs in the side region will be decided later.
MC studies in this proposal is performed assuming that one MRD is placed in the side region.
% and each detector consists of a sandwich structure of iron plates and tracking scintillator planes.
% One of four $\mu$ range detectors in the side region is shown in the Fig. \ref{fig:all_detector}.

\begin{figure}[tbh]
\begin{center}
\includegraphics[width=0.8\linewidth]{fig/all_detector2.pdf}
\end{center}
\caption{
Schematic view of entire sets of detectors.
}
\label{fig:all_detector}
\end{figure}


The dimension of the central detector is 100cm $\times$ 100cm in the x and y directions
and 200cm along the beam direction.
The total water and hydrocarbon masses serving as neutrino targets are $\sim$1 ton each.
Inside the central detector, plastic scintillator bars are aligned as a 3D grid-like structure, shown in Fig. \ref{fig:3dgrid},
and spaces in the structure are filled with the neutrino target materials, water and hydrocarbon.
When neutrinos interact with hydrogen, oxygen or carbon, in water and hydrocarbon,
charged particles are generated.
Neutrino interactions are identified by detecting tracks of charged particles through plastic scintillation bars.
Thanks to the 3 D grid-like structure of the scintillator bars, 
the central detector has $4\pi$ angular acceptance for charged particles.
Furthermore, adopting a 2.5cm grid spacing, short tracks originated from protons and charged pions can be reconstructed
with high efficiency.
Thin plastic scintillator bars (thickness $\sim 0.3$cm) will be used for the central detector
to reduce  the mass ratio of scintillator bars to neutrino target materials,
because neutrino interactions in the scintillator bars are a background for the cross section measurements.
Scintillator bars whose dimensions are 2.5cm x 0.3cm x 100cm will be used for the central detector.
The total number of channels in the central detector is 12880.

\begin{figure}[tbh]
\begin{center}
\includegraphics[width=0.6\linewidth]{fig/3d_grid_detector.pdf}
\end{center}
\caption{
Schematic view of 3D grid-like structure of plastic scintillator bars inside the central detector.
}
\label{fig:3dgrid}
\end{figure}


The dimension of the MRD in the side (downstream) region is
$\sim$200(200)cm $\times$ $\sim$300(350)cm in a plane perpendicular to the muon direction
and $\sim$75(230)cm along the muon direction.
The MRD in the side (downstream) region consists of 12(30) 3 cm thick iron plates and 14(32) tracking scintillator planes.
Muons generated at smaller angle to the beam axis have higher momentum,
so the MRD in the downstream region is thicker along the muon direction.
Each tracking scintillator layer of the MRD in the side (downstream) region
has 25(28) scintillator bars whose dimensions are 20cm x 1cm x 200(200)cm and 20cm x 1cm x 300(350)cm,
making a plane measuring 300(350)$\times$200(200)cm$^{2}$ in the horizontal and vertical directions and 2 cm along the muon direction.
The total number of channels in the MRD is 1246.
The role of the MRDs is the selection of muon tracks from the charged-current (CC) interactions
and the rejection of short tracks caused by neutral particles 
%from outside the central detector,
that originate mainly from neutrino interactions in material surrounding the central detector, like the walls of the detector hall,
neutrons and gammas, or neutral-current (NC) interactions.
The muon momentum can be reconstructed from its range inside the detector.
The MRDs are located 50cm away from the central detector
to identify the direction of motion of charged particles from the hit-time difference between the two detectors,
and reject charged-particle background
that originates from neutrino interactions in the material surrounding the central detector, like the walls of the detector hall and the MRDs themselves.
% from outside the central detector.
% which originate from neutrino interactions in the walls of the experimental hall and the $\mu$ range detector.


Scintillation light in the scintillator bar is collected and transported to a photodetector with a wavelength shifting fiber (WLS fiber).
The light is read out by a photodetector, Multi-Pixel Photon Counter (MPPC), attached to one end of the WLS fiber.
The signal from the MPPC is read out by the dedicated electronics developed for the test experiment
%developed by our group 
to enable bunch separation in the beam spill.
The readout electronics is triggered using the beam-timing signal from MR to synchronize to the beam.
The beam-timing signal is branched from those for T2K, and will not cause any effect on the T2K data taking.


T2K is adopting the off-axis beam method, in which
the neutrino beam is directed 2.5 degrees away from SK producing a narrowband $\nu_{\mu}$ beam.
The off-axis near detector, ND280, is installed towards the SK direction in the B1 floor of the near detector hall of the J-PARC neutrino beam-line.
We are planning to install our detector in the B2 floor of the near detector hall, 
where the off-axis angle is similar, and therefore an energy spectrum similar to ND280 and SK is expected.
The candidate detector position in the B2 floor is shown in Fig. \ref{fig:location}.
The expected neutrino energy spectrum at the candidate position is shown in Fig. \ref{fig:b2flux}.

\begin{figure}[tbh]
\begin{center}
\includegraphics[width=0.6\linewidth]{fig/location2.pdf}
\end{center}
\caption{
Candidate detector position in the B2 floor of the near detector hall.
}
\label{fig:location}
\end{figure}

\begin{figure}[tbh]
\begin{center}
\includegraphics[width=0.6\linewidth]{fig/b2flux2.pdf}
\end{center}
\caption{
Neutrino energy spectrum at the candidate detector position(red).
The spectrum at the ND280 site (black) is also shown.
}
\label{fig:b2flux}
\end{figure}
