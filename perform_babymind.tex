\subsection{Baby MIND}
The Baby MIND construction was completed in June 2017, and it was then tested in June and July 2017 at the Proton Synchrotron experimental hall at CERN with a mixed particle beam comprising mostly muons whose momenta could be selected between 0.5 and 5 GeV/c. An event display from the summer 2017 tests is shown in Figure \ref{beam_vs_cosmic}.

\inserttwographs{0.45}{baby_mind_Wagasci_rough_cad.png}{0.50}{baby_mind_layout.png}{Left) WAGASCI modules: flanked by 2 side muon range detectors (sMRD) and one downstream muon detector (Baby MIND). Right) side view layout of the Baby MIND during beam tests at CERN.}{wagasci-layout}


\insertgraph{1.0}{comparing_beam_cosmics.png}{Comparison of a beam muon and cosmic muon in the three different geometrical projections of the detector. The beam impinges on the detector from the left. The arrows indicate the direction of travel of these muons. The direction of the cosmic muon is inferred from timing information.}{beam_vs_cosmic}


All counters were measured at INR Moscow with a cosmic ray setup using the same type S12571-025C MPPCs and CAEN DT5742 digitizer \cite{Antonova:2017cdw}. The average light yield (sum from both ends) was measured to be 37.5 photo-electrons (p.e.) per minimum ionizing particle (MIP) and 65 p.e./MIP for vertical and horizontal counters, respectively. After shipment to CERN, all counters were tested once more individually with an LED test setup \cite{led_test_system}. 0.1\% of counters failed the LED tests and were therefore not used during the assembly of modules. 
