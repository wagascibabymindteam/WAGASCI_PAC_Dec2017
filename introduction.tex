\section{Introduction}

The understanding of neutrino-nucleus interactions in the 1 GeV energy region is critical for the success
of accelerator-based neutrino oscillation experiments such as the T2K experiment.
Complicated multi-body effects of nuclei render this understanding difficult.
The T2K near detectors have been measuring these and significant progress has been achieved.
However, the understanding is still limited.
One of the big factors preventing a complete understanding is the non-monochromatic neutrino beam spectrum.
Measurements with distinct but partially overlapping beam spectra would be a great benefit
in resolving the contribution from different neutrino energies.
We, the WAGASCI collaboration, proposes to study the neutrino-nucleus interaction
at the B2 floor of the neutrino monitor building, where different neutrino spectra from the T2K off-axis near detector (ND280) can be obtained due to the different off-axis position.
Our experimental setup contains two hollow cuboid lattice detectors filled with water as the neutrino interaction target (known as WAGASCI modules), two side- and one downstream- muon range detectors(MRD's).
We will have two types of the WAGASCI modules, a water-in module and a water-out module.
The water-in WAGASCI module has water the hollow cuboid lattice, and the water-out WAGASCI module doesn't have water inside the lattice.
The hollow cuboid lattice and side-MRD's allow a measurement of  wider-angle scattering than ND280.
High water to scintillator material ratio enables the measurement of the neutrino interaction with water, which is highly desired for the T2K experiment because it's far detector, Super-Kamiokande, is composed of water.
The MRD's consist of plastic scintillators and iron plates.
The downstream-MRD, so called the Baby MIND detector, also works as a magnet and provides the charge identification capability as well as magnetic momentum measurement for high energy muons.
The charge identification is essentially important to select antineutrino events in the antineutrino beam
because contamination of the neutrino events is as high as 30\%.
Most of the detectors have already been constructed.
The WAGASCI modules have been commissioned as the J-PARC T59 experiment and the Baby MIND detector was commissioned at the CERN neutrino platform.
Therefore, the collaboration will be ready to proceed to the physics data taking by January 2019.
We will provide the cross sections of the charged current neutrino and antineutrino interactions on water
with slightly higher neutrino energy than T2K ND280 with wide angler acceptance.
The requested beam time is one-year in neutrino-mode and another one-year in antineutrino mode
assuming that the POT for the fast extraction in each year is more than $5\times10^{20}$~POT.
When combined with ND280 measurements, our measurement would greatly improve the understanding of the neutrino interaction
at around 1 GeV 
and contribute to reducing one of the most significant uncertainties of the T2K experiment.

