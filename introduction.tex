\section{Introduction}

The understanding of neutrino-nucleus interactions in the 1 GeV energy region is critical for the success
of accelerator-based neutrino oscillation experiments such as the T2K experiment.
Complicated multi-body effects of nuclei render this understanding difficult.
The T2K near detectors have been measuring these and significant progress has been achieved.
However, the understanding is still limited.
One of the big factors preventing from full understanding is the non-monochromatic
neutrino beam spectrum.
Measurements with different but some overlapping beam spectra would greatly benefit to resolve the contribution
from different neutrino energies.
We, the Wagasci collaboration, proposes to study the neutrino-nucleus interaction
at the B2 floor of the neutrino monitor building, where different neutrino spectra
can be obtained due to different off-axis positions.
Our experimental setup contains 3D grid-structure plastic-scintillator detectors filled with water as the neturino interaction target
(Wagasci modules), two side- and one downstream- muon range detectors(MRD's).
The 3D grid-structure and side-MRD's allows a measuremen of  wider-angle scattering than the T2K off-axis near detector (ND280).
High water to scitillator material ratio enables the measurement of the neutrino interaction on water, which
is higly desired for the T2K experiment because it's far detector, Super-Kamiokande, is composed of water.
The MRD's consist of plastic scintillators and iron plates.
The downstream-MRD, so called the Baby MIND detector, is also work as a magnet and provides the charge identification capability.
The charge identification is essentially important to select antineutrino events in the antineutrino beam
because contamination of the neutrino events is as high as 30\%.
Most of the detectors has been already constructed and commissioned as the J-PARC T59 experiment.
Therefore, the collaboration will be ready to proceed to the physics data daking for the T2K beam time in January 2019.
We will provide the cross sections of the charged current neutrino and antineutrino interactions on water
with slightly higher neutrino energy than T2K ND280 with wide angler acceptance.
When combined with ND280 measurements, our measurement would greatly improve the understanding of the neutrino interaction
at around 1 GeV and contribute to reduce the most significant uncertainty of the T2K experiment.
