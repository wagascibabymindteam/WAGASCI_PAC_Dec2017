\subsection{Baby MIND}
The Baby MIND is the downstream Muon Range Detector.
It also works as a magnet and provides the charge identification capability
as well as magnetic momentum measurement for high energy muons.

The Baby MIND collaboration \footnote{Contact person: E. Noah, Spokesperson: A. Blondel, Deputy spokesperson: Y. Kudenko} submitted a proposal to the SPSC at CERN, SPSC-P-353.
The project was approved by the CERN research board as Neutrino Platform project NP05 and constructed.
The detector consists of 33 magnet modules, each 3500 mm $\times$ 2000 mm$ \times$ 50 mm (30 mm steel) with a mass of approximately 2 tonnes. Of these magnet modules, 18 are instrumented with plastic scintillator modules. 

%The main Baby MIND systems are the magnet, scintillator and electronics modules \cite{Noah:EPS2017}.
%A total of 3,996 silicon photomultipliers are read out by custom electronics Front End Boards that can process up to 96 channels each, sending charge and timing information of hits in the detector to dedicated data acquisition computers.

%One challenge to be addressed by the Baby MIND collaboration is that of obtaining high charge identification efficiencies for $\mu^+/\mu^-$ down to 500 MeV/c and below. Magnetized iron neutrino detectors are limited by multiple scattering in the iron, and their use is overlooked for applications requiring good charge ID efficiencies below 1 GeV/c. By optimizing the distance between the first magnet modules, rendered possible by the magnet design, our simulations show improved charge identification efficiencies down to 400 MeV/c.


\subsubsection{Magnet modules}
Traditional layouts for magnetized iron neutrino detectors (e.g. MINOS) tend to be monolithic blocks
with a unique pitch between consecutive steel segments and large conductor coils threaded around the whole magnet volume.
The Baby MIND detector, like traditional designs, is built from sheets of iron interleaved with scintillator detector modules.
However Baby MIND is novel in that the iron segments are all individually magnetized as shown in Figure ~\ref{fig:BM_winding}, allowing for far greater flexibility in the setting of the pitch between segments, and in the allowable geometries that these detectors can take.


The key design outcome is a highly optimized magnetic field map. A double-slit configuration for coil winding was adopted to increase the area over which the magnetic flux lines are homogeneous in $B_x$ across the central tracking region. Simulations show the magnet field map to be very uniform over this central tracking region covering an area of $2800\times2000$ mm$^2$, Figure~\ref{magnet-cross-sections}. The $B_x$ component dominates in this region, with negligible $B_y$ and $B_z$. This was confirmed by measuring the field with 9 pick-up coils wound around the first module. Subsequent modules were equipped with one pick-up coil. Test results on the 33 modules show all achieve the required field of 1.5 T for a current of 140 A, with a total power consumption of 11.5 kW. The polarity of the field map shown in Figure~\ref{magnet-cross-sections} (middle) can be reversed by changing the power supply configuration.
\insertgraph{0.5}{magnet_assembly_zone.pdf}{Magnet assembly zone at CERN. }{fig:BM_winding}
\inserttwographs{0.5}{b-field-1500a-280cm-coil.pdf}{0.4}{magnetic_field_measurements_per_batch.pdf}{Left) Magnetic field map with a coil along 2800 mm of the length of the plate. Right) Measured B field for 33 modules.}{magnet-cross-sections}

\subsubsection{Scintillator modules}
Each of the 18 scintillator modules is constructed from 2 planes of horizontal counters (95 counters in total) and 2 planes of vertical counters (16 counters in total) \cite{Antonova:2017tuf}, arranged with an overlap between planes to achieve close to 100\% hit efficiency for minimum ionizing muons. The arrangement of planes within a module is vertical-horizontal-horizontal-vertical. This arrangement was the result of an assembly approach whereby each plane was built from 2 half-planes, with each half plane consisting of a horizontal plane and a vertical plane.
The scintillator bars are held in place using structural ladders that align and maintain the counters, Figure~\ref{proto-module-ladders}. No glue is used in the process, so counters can be replaced. Aluminum sheets front and back provide light tightness.

\insertthreegraphs{0.32}{top_front_module.pdf}{0.32}{bottom_front_module.pdf}{0.32}{rear-half-module.pdf} {Scintillator modules assembly. Left) top of front half-module showing vertical counters, and the spacer-ladders that set the pitch between horizontal counters and hold them in place. Middle) rear half-module showing horizontal counters on their ladders. Right) Assembled rear half-module, the front half-module can be seen in the background.}{proto-module-ladders}

The plastic scintillator counters were made from 220 mm-wide slabs, consisting of extruded polystyrene doped with 1.5\% paraterphenyl (PTP) and 0.01\% POPOP. They were cut to size then covered with a 30-100 $\mu$m thick diffuse reflector resulting from etching of the surface with a chemical agent \cite{Kudenko:2001qj, Mineev:2011xp}. The horizontal counter size is $2880 \times 31 \times 7.5 $ mm$^3$, with one groove along the length of the bar in which sits a wavelength shifting fiber from Kuraray. The vertical counter size is $1950 \times 210 \times 7.5 $ mm$^3$, with one U-shaped groove along the bar. On each counter, two custom connectors house silicon photomultipliers, MPPC type S12571-025C from Hamamatsu, either side of the horizontal counter, and both connectors at the top for the vertical counter. This geometrical configuration for vertical counters was chosen for ease of connectivity to the electronics, and maintenance operations.

A total of 1744 horizontal counters and 315 vertical counters (including spares) were produced at the Uniplast company (Vladimir, Russia).

\subsubsection{Electronics}
The Baby MIND electronic readout scheme includes several custom-designed boards \cite{Noah:2016ikh}. The revised version is shown in Figure~\ref{block_diagrams}. At the heart of the system is the electronics Front End Board (FEB), developed by the University of Geneva. The readout system includes two ancillary boards, the Backplane, and the Master Clock Board (MCB) whose development has been managed by INRNE (Bulgarian Academy of Sciences) collaborators.

%One critical element in the photosensor readout path is the cable bundle, a 5 m extension coaxial cable RG174U that connects the photosensor to the FEB. Each bundle connects up to 32 photosensors. The purpose is to decouple the FEBs from the scintillator modules, which improves accessibility to FEBs and their long term maintainability. The module end of the bundle hosts some electronics that manages the application of the high voltage to the SiPMs, enabling faulty SiPMs to be switched off at that level. This feature was added after the summer 2016 beam tests, where a short circuit on a single channel would disable a bank of 96 channels.

\inserttwographs{0.47}{readout_scheme_2.pdf}{0.45}{electronics_connectivity_detailed.pdf}{Left) Baby MIND electronics readout scheme. Right) SiPM-to-FEB connectivity.}{block_diagrams}

The FEBv2 hosts 3 CITIROC chips that can each read in signals from 32 SiPMs \cite{Fleury:2014hfa}.
Each signal input is processed by a high gain (HG), and a separate low gain (LG), signal path.
The outputs from the slow shapers can be sampled using one of two modes: a mode with an externally applied delay, and a peak detector mode. A faster shaper can be switched to either HG or LG paths, followed by discriminators with adjustable thresholds providing 32 individual trigger outputs and one OR32 trigger output. An Altera ARIA5 FPGA on the FEBv2 samples these trigger outputs at 400 MHz, recording rising and falling times for the individual triggers and assigning time stamps to these. 
Time-over-threshold, the difference between falling and rising times, gives some measure of signal amplitude. This is used in addition to charge information and proves useful if there is more than one hit per bar within the $\sim9$ $\mu$s deadtime due to the readout of the multiplexed charge output.
The ARIA5 also manages the digitization of the sampled CITIROC multiplexed HG and LG outputs via a 12-bit 8-ch ADC. 

%The FEBv2 is designed to fit into a slot in a minicrate as shown in Figure \ref{minicrate}. The front face receives the SiPMs cable bundles, the rear end plugs into the backplane. Up to 6 FEBv2 can be housed in each minicrate. Eight minicrates are distributed either side of the Baby MIND.

The internal 400 MHz clock on the FEBv2 can be synchronized to a common 100 MHz clock. The synchronization subsystem combines input signals from the beam line into a digital synchronization signal (SYNC) and produces a common detector clock (CLK) which can eventually be synchronised to an external experiment clock. Both SYNC and CLK signals are distributed to the FEBs. Tests show the FEB-to-FEB CLK(SYNC) delay difference to be 50 ps (70 ps). Signals from the beam line at WAGASCI include two separate timing signals, arriving 100 ms and 30 $\mu$s before the neutrino beam at the near detectors. The spill number is available as a 16-bit signal. 
%\insertthreegraphs{0.31}{FEBv2_horizontal_low_res.jpg}{0.2}{FEB_v2_in_crate_low_res2.jpg}{0.145}{rear_minicrate.jpg}{Left) Second version of the electronics readout Front End Board (FEBv2), received at the University of Geneva in March 2017. Middle) FEBv2 in minicrate. Right) rear of minicrate with 6 FEBs connected through a Backplane PCB on the lower half of the minicrate.}{minicrate}

%Several connection options are possible between the FEBv2 and a DAQ PC. The FEBv2 can be operated as a standalone device, connected directly to a DAQ PC via USB3. This is useful for laboratory measurements on the FEB itself, its maintenance and calibration, and qualification tests on other components such as MPPCs or cable bundles. It is also possible to daisy chain several FEBs via the backplane PCB in experiment data taking mode, with the first FEB in the chain connected directly to the DAQ PC via a USB3 link. In this mode the USB3 bandwidth is shared with the potential 6 FEBs in the chain thanks to a Time Division Multiplexing (TDM) protocol, each FEB having $1/6$th of the data throughput. For enhanced measurements or calibration a dedicated option of the chaining is also possible where 1 single FEB in the chain can use the full bandwidth of the single USB3 connection. The DAQ software is platform independent. The data protocol encodes information such as spill number, FEB ID, hit channel number, time and charge, as well as tags to match the TDM data stream to the correct minicrate slot ID.

\subsubsection{Pefromance check}
All counters were measured at INR Moscow with a cosmic ray setup using the same type S12571-025C MPPCs and a CAEN DT5742 digitizer.% \cite{Antonova:2017cdw}.
The average light yield (sum from both ends) was measured to be 37.5 photo-electrons (p.e.) per minimum ionizing particle (MIP) and 65 p.e./MIP for vertical and horizontal counters, respectively. 
After shipment to CERN, all counters were individually re-tested with an LED \cite{led_test_system}. 
0.1\% of counters failed the LED tests and were therefore not used during the assembly of modules. 
The assembly of modules was completed in June 2017, and it was then tested in June and July 2017 at the Proton Synchrotron experimental hall at CERN with a mixed particle beam comprising mostly muons whose momenta could be selected between 0.5 and 5 GeV/c. An event display from the summer 2017 tests is shown in Figure~\ref{beam_vs_cosmic}.
%\inserttwographs{0.45}{baby_mind_wagasci_rough_cad.pdf}{0.50}{baby_mind_layout.pdf}{Left) WAGASCI modules: flanked by 2 side muon range detectors (sMRD) and one downstream muon detector (Baby MIND). Right) side view layout of the Baby MIND during beam tests at CERN.}{wagasci-layout}
\insertgraph{1.0}{comparing_beam_cosmics.pdf}{Comparison of a beam muon and cosmic muon in the three different geometrical projections of the detector. The beam impinges on the detector from the left. The arrows indicate the direction of travel of these muons. The direction of the cosmic muon is inferred from timing information.}{beam_vs_cosmic}

