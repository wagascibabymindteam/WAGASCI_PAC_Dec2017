\section{Physics goals}
We will measure the differential cross section for the charged current interaction on $\mathrm{H_2O}$ and/or CH.
The water-scntillator mass ratio of the Wagasci module is as high as 5:1 and the high purity measurement
of the cross section on $\mathrm{H_2O}$ is possible.
\textcolor{red}{One experimental option is to replace one of the two Wagasci module with the T2K proton module
  which is fully made with plastic scintillators. It will allow the precise comparison
  of cross section between $\mathrm{H_2O}$ and CH and also comparison of cross sections with ND280.}
  \textcolor{red}{Another option is to remove water from one of the two Wagasci module. 
The water-out WAGASCI module will make it possible to measure wider- angle scatterings for CH target and will provide a low density medium for the detection of low momentum protons.
The water-out WAGASCI data also can be used to subtract the background from interaction with scintillators in the water target measurement .
}
Our setup would allow the measuemrents of inclusive and also exclusive channles such as
1-$\mu$, 1-$\mu 1p$, 1-$\mu 1\pi{\pm} np$ samples, former two of which are mainly caused by the quasi-elastic and
2p2h interaction and the latter is mainly caused by resonant or coherent pion production and deep elastic scattering.
In general, an accelerator produced neutrino beam spectrum is wide and the energy reconstruction
somehow rely on the neutrino interaction model.
Therefore, recent neutrino cross section measurement results including T2K are given as a flux-integrated cross section
rather than cross sections as a function of the neutrino energy to avoid the model dependency.
We can provide the flux-averaged cross section.
In addition, by combining our measurements with those at ND280, model-independent extraction of the cross section
for narrow energy region becomes possible.
This method was demonstrated in \ref{ingrid_energy_dependent} and also proposed by P** (NUPRISM).
\textcolor{red}{add Yasutome plot here or later.}


\subsection{Expected number of events}
Expected number of neutrino events after the event selections is evaluated with Monte Carlo simulations as we will discuss in Section \ref{sec:mc_study}.
$2.41 \times 10^{4}$ CC events are expected in two WAGASCI modules after the selection with $1\times 10^{20}$ POT in neutrino-mode, and its purity is 75.5 \%.
In case we choose the option with one WAGASCI module and the T2K proton module,  $1.2 \times 10^{4}$ CC events are expected in the WAGASCI module and $\sim 1\times 10^{4}$ CC events are expected in the T2K proton module.
In case we choose the option with one water-in WAGASCI module and one water-out WAGASCI module,  $1.2 \times 10^{4}$ CC events are expected in the water-in module and $0.24 \times 10^{4}$ CC events are expected in the water-out module.


\subsection{Nuclear effects}
In T2K experiment, neutrinos interact with bound nucleons in relatively heavy nuclei (Carbon and Oxygen), so the cross-section is largely affected by nuclear effects.
The nuclear effects are categorized as nucleons' momentum distribution in nucleus, interactions with  correlated pairs of nucleons in nucleus (two particles-two holes, 2p2h), corrections from collective nuclear effects calculated with Random Phase Approximation (RPA) and final state interactions (FSI) of secondary particles in the nuclei after the initial neutrino interactions.





The 2p2h interactions mainly happen through $\Delta$ resonance interactions following a pion-less decay and interactions with a correlated nucleon pair.
The 2p2h interactions are observed in electron scattering experiments (add ref. here) where the 2p2h events observed in the gap between Quasi-Elastic region and Pion-production region as shown in Fig. \ref{fig:electrono_scattering_data}.
Neutrino experiments also attempt to measure the 2p2h interactions, but separation of the QE peak and the 2p2h peak is more difficult because transferred momentum (p) and energy (w) are largely affected by  neutrino energy which cannot be determined event-by-event in the wide energy spectrum of the accelerator neutrino beam.
Our model-independent narrow neutrino spectra extracted from combined analyses of our data and ND280 data are ideal for searching the 2p2h interaction because clearer separation of the QE peak and the 2p2h peak is expected.


The corrections from collective nuclear effects calculated by RPA as a function of $Q^{2}$ are shown in Fig. \ref{fig:effect_rpa}.
The $Q^{2}$ dependence of the correction can be tested by measuring angular distribution of muons in CC1-$\mu$ and CC1-$\mu 1p$ events.
The uncertainties of the corrections in low (high) $Q^{2}$ regions can be constrained by observing the events with a forward-going (high-angle) muon, so it is essential to measure muon tracks with full acceptance.


T2K experiment is starting to use $\nu_{e}$ CC1$\pi$ events for its CP violation search to increase the statistics.
One of the biggest uncertainty of CC1$\pi$ sample comes from the final state interactions of pions in the nuclei after the initial neutrino interactions because they change the multiplicity, charge and kinematics of the pions.
The multi-pion production events can be migrated into the CC1$\pi$ sample due to the FSIs, and they become important backgrounds.
We can constrain the uncertainties from the pion FSIs by measuring pion rescattering in the detector and pion multiplicity in CCn$\pi$ sample with low detection threshold and full acceptance for pion tracks.
The water-out WAGASCI can provide good sample for the pion FSI studies because its low density medium enables the detection of low momentum pions in addition to the full acceptance.

